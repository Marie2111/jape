\chapter{The command language, environment variables and the default environment}
\label{appx:GUIlang}


\section{The command language}


Jape communicates with its graphical interface in a language of `words', space separated unless they are enquoted "...". You may want to attach commands to buttons, you may want to include commands as entries in \textsc{tacticpanel}s, and you can type commands into the system --- on the Mac into the Text Command box, on X into the command window --- so here goes with a description. I've divided it into two: the ones you might want to use, and the arcana.


Note that the language described here is \textit{ad hoc} and subject to change without notice or any sign of visible regret on the part of the implementors. Be warned.


\subsection{Commands you might want to use}


addnewconjecture \textit{panelname} \textit{conjecture}: this command is sent by the New\dots  button in a conjecture panel, after the user has typed the conjecture into a dialogue box.

apply \textit{tacticexpression}: this command is used a lot: it is the way that menus and panels apply tactics. Don't forget that a rule name is a tactic expression.


assign \textit{name} \textit{value}: the way that Jape's environment variables (q.v. below) are given values.


backtrack: command sent by the Backtrack button in the Edit menu.


closedbugfile: close the top dbug file on the stack of such files (see createdbugfile below); redirect diagnostic output to the file below, or to the console if the stack is empty.


collapse: the way that the Hide/Show detail entry in the Edit menu does its work.


createdbugfile: create a file, using the normal file selection dialogue, and redirect diagnostic input into it. There's a stack of these dbug files.


layout: has the same effect as double-clicking on the justification of the selected sequent.

lemma \textit{panelname} \textit{conjecture}: synonym for addnewconjecture above.


print \textit{filename}: generates a listing of the currently-focussed proof in \textit{filename}, in a form suitable for LaTeX processing.

proof finished: (two words) how the Done entry (on the Mac) and the proof finished entry (on UNIX) in the Edit menu does its work.


prove \textit{conjecturename}: how the Prove button in a conjecture panel does its work.

prune: how the Prune entry in the Edit menu does its work.

QUIT: kill the proof engine, after asking whether the user wants to save any proofs.

redo: how the Redo entry in the Edit menu does its work.

refreshdisplay: clear the currently-focussed proof window and redraw the proof it contains.


reset: how the Reset entry (on the Mac) and the ?? entry (on UNIX) in the Edit menu do their work. On the Mac the Reset entry can be greyed-out even though some syntax definitions have been accepted: in that case typing the reset command to a Text Command window can be helpful.


reset;reload: (no spaces, all one `word' with a semicolon in the middle!) how the Load New Theory entry in the File menu does its work.

showproof \textit{conjecturename}: how the Show Proof button on a conjecture panel does its work; opens a window with a proof of \textit{conjecturename} in it, if there is one in the proof store.

saveengine \textit{filename}: saves the current proof engine, with all its settings, in a file. Useful for creating pre-initialised engines

steps: display the value of the internal variable `timestotry' in an alert dialogue. See steps \textit{n} below.


steps \textit{n}: set the value of the internal variable `timestotry' to the integer \textit{n}. This variable will, in the near future, be part of the default environment (q.v. below). The value of the variable controls the number of steps that Jape will allow in a single tactic application before failing with the message ``Time ran out''.


tellinterface \textit{variablename word word....}: send the current value of the variable \textit{variablename} to the interface, prefixed by the command \textit{word....}.


undo: the way that the Undo entry in the Edit menu and/or the Undo key do their work.


unify \textit{formulae}: Unify the given formulae and all of the user's text-selections. The way that the Unify button does its work.


use \textit{filenames}: open each of the files named, read and execute the Japeish text they contain. The way that proof files are loaded and a new encodings or a modification to the current encoding is interpreted


version: display the current version information of the Jape engine in an alert dialogue.


\textit{Arcana}


cd \textit{path}: changes the default directory used by the proof engine. Only works in the UNIX implementations; don't use if you don't know what it does.


closeproof \textit{n}: absolutely not to be used.

DRAGQUERY: part of the drag-and-drop interface; don't use it.

DROPCOMMAND: part of the drag-and-drop interface; don't use it.


fonts\_reset: command sent by the graphics interface when its fonts are altered. Triggers all sorts of cache mangling, but otherwise harmless.


HITCOMMAND \textit{comm}: absolutely not to be used.

NOHITCOMMAND \textit{comm}: absolutely not to be used.


profile [ on {\textbar} off {\textbar} reset {\textbar} report \textit{filename} ]: one of the mechanisms with which we debug the proof engine. Only works in specially-instrumented proof engines under \textsc{unix}.


quit: kill the proof engine without asking any questions.

saveproofs \textit{word}: absolutely not to be used.

setfocus \textit{n}: absolutely not to be used.

showfile \textit{filename}: possibly obsolete.

\textit{emptyword}: ignored.


\section{Variables and the default environment}


Jape has a number of `environment variables' which can be used to modify its behaviour, and can currently be set by the \textsc{assign} tactic, by the assign command and by \textsc{initialise, radiobutton} and \textsc{checkbox} directives in the paragraph language. Some of them are of general use; some are horrid debugging switches of interest only to the implementors. Variables can be set from menus and panels: see the various files like `autoselect\_entry' which are distributed with Jape and put entries in the Edit menu.


\subsection{Useful variables}


Variables whose default value is marked with an asterisk are parameters: their value can be altered only if the rule/tactic/conjecture store is empty.



\begin{tabular}{|p{1.034in}|p{0.635in}|p{0.717in}|p{2.114in}|} \hline
% ROW 1
{\raggedright \textit{name}} & {\raggedright \textit{values}} & {\raggedright \textit{default value}} & {\raggedright \textit{effect}}\\
\hline
% ROW 2
{\raggedright applyconjectures} & {\raggedright true, false} & {\raggedright false} & {\raggedright when true, allow conjectures (unproved theorems) to be applied as rules.}\\
\hline
% ROW 3
{\raggedright autoAdditiveLeft} & {\raggedright true, false} & {\raggedright false*} & {\raggedright when true, any rule whose consequent and antecedents all have a bag on their left-hand sides is augmented by the addition of a bag variable (e.g. \ensuremath{\Gamma}) to the left-hand side of every consequent and antecedent which doesn't already have one.}\\
\hline
% ROW 4
{\raggedright autoAdditiveRight} & {\raggedright true, false} & {\raggedright false*} & {\raggedright as autoAdditiveLeft, except that it applies to right-hand sides}\\
\hline
% ROW 5
{\raggedright autoselect} & {\raggedright true, false} & {\raggedright false} & {\raggedright when true, select the conclusion of the current problem sequent each time a proof is displayed.}\\
\hline
% ROW 6
{\raggedright collapsedfmt} & {\raggedright any string} & {\raggedright "[\%s...]"} & {\raggedright the string used to control the way that a justification is displayed for a subtree shown in `collapsed' form --- for example, after using Hide/Show Subproof on an uncollapsed subtree.}\\
\hline
% ROW 7
{\raggedright displaystyle} & {\raggedright box, tree} & {\raggedright tree} & {\raggedright selects the display mechanism used to show a proof. Each proof may have an individual setting of this variable. When a new proof is started, its displaystyle is taken from the currently-focussed proof.}\\
\hline
% ROW 8
{\raggedright hiddenfmt} & {\raggedright any string} & {\raggedright "\{\%s\}"} & {\raggedright the string used to control the way that a justification is displayed for a subtree produced by the layout tactical in `hidden' form. This string is over-ridden if \textit{string1} is provided in the layout tactical.}\\
\hline
% ROW 9
{\raggedright hidecut} & {\raggedright true, false} & {\raggedright true} & {\raggedright hide the application of cut rules in box display.}\\
\hline
% ROW 10
{\raggedright hidehyp} & {\raggedright true, false} & {\raggedright true} & {\raggedright hide the application of identity rules in box display.}\\
\hline
% ROW 11
{\raggedright interpretpredicates} & {\raggedright true, false} & {\raggedright false*} & {\raggedright on instantiating a rule, interpret juxtapositions as predicate applications; translate them into substitution forms, add new object parameters and invisible provisos.}\\
\hline
% ROW 12
{\raggedright outermostbox} & {\raggedright true, false} & {\raggedright true} & {\raggedright when true, draw an outermost box in box display when proving a conjecture which has hypothesis formulae.}\\
\hline
% ROW 13
{\raggedright showallproofsteps} & {\raggedright true, false} & {\raggedright true} & {\raggedright (misnamed --- should be showhiddenproofsteps) when true, show proof steps hidden by layout tacticals.}\\
\hline
% ROW 14
{\raggedright showallprovisos} & {\raggedright true, false} & {\raggedright false} & {\raggedright (misnamed --- should be showhiddenprovisos) when true, show hidden provisos, marked as \texttt{<}proviso\texttt{>}.}\\
\hline
% ROW 15
{\raggedright tryresolution} & {\raggedright true, false} & {\raggedright true} & {\raggedright apply theorems and antecedent-free rules in `resolution' style if the conclusions of the consequent match but the hypotheses don't.}\\
\hline
% ROW 16
{\raggedright uncollapsedfmt} & {\raggedright any string} & {\raggedright "\%s"} & {\raggedright string that controls the display of a subtree that was once collapsed and is now reinflated.}\\
\hline
% ROW 17
{\raggedright unhiddenfmt} & {\raggedright any string} & {\raggedright "[\%s]"} & {\raggedright string that controls the display of a subtree produced by the layout tactical and displayed in `normal' form. This string is over-ridden if \textit{string2} is provided in the layout tactical.}\\
\hline \end{tabular}


\subsection{Adding your own variables}


You can invent your own environment variables and assign them values. In particular you can define a variable in a \textsc{radiobutton} or \textsc{checkbox} directive, give the range of possible values that it can take, and allow the user to control that variable. See, for example, the way that the functional programming encoding controls searching by using variables whose values are the names of tactics.


There are at present few ways in which the value of a variable can be used, once set. But watch this space for developments, including at least a form of case-expression value analysis in the tactic language.


\subsection{Debugging variables}


Jape has a number of debugging variables. Setting any of them to true makes it print lots of stuff on the console (which on the Mac is hidden, and needs secret knowledge to find). The variables currently used are


applydebug, bindingdebug, factsdebug, FINDdebug, FOLDdebug, matchdebug, prooftreedebug, rewritedebug, substdebug, symboldebug, tactictracing, thingdebug, unifydebug, eqalphadebug, varbindingsdebug


In this manual we don't explain or admit what these variables do or don't do or how best to use them. Good luck to you if you try to find out.
 